% Options for packages loaded elsewhere
\PassOptionsToPackage{unicode}{hyperref}
\PassOptionsToPackage{hyphens}{url}
%
\documentclass[
]{article}
\usepackage{amsmath,amssymb}
\usepackage{iftex}
\ifPDFTeX
  \usepackage[T1]{fontenc}
  \usepackage[utf8]{inputenc}
  \usepackage{textcomp} % provide euro and other symbols
\else % if luatex or xetex
  \usepackage{unicode-math} % this also loads fontspec
  \defaultfontfeatures{Scale=MatchLowercase}
  \defaultfontfeatures[\rmfamily]{Ligatures=TeX,Scale=1}
\fi
\usepackage{lmodern}
\ifPDFTeX\else
  % xetex/luatex font selection
\fi
% Use upquote if available, for straight quotes in verbatim environments
\IfFileExists{upquote.sty}{\usepackage{upquote}}{}
\IfFileExists{microtype.sty}{% use microtype if available
  \usepackage[]{microtype}
  \UseMicrotypeSet[protrusion]{basicmath} % disable protrusion for tt fonts
}{}
\makeatletter
\@ifundefined{KOMAClassName}{% if non-KOMA class
  \IfFileExists{parskip.sty}{%
    \usepackage{parskip}
  }{% else
    \setlength{\parindent}{0pt}
    \setlength{\parskip}{6pt plus 2pt minus 1pt}}
}{% if KOMA class
  \KOMAoptions{parskip=half}}
\makeatother
\usepackage{xcolor}
\usepackage[margin=1in]{geometry}
\usepackage{graphicx}
\makeatletter
\def\maxwidth{\ifdim\Gin@nat@width>\linewidth\linewidth\else\Gin@nat@width\fi}
\def\maxheight{\ifdim\Gin@nat@height>\textheight\textheight\else\Gin@nat@height\fi}
\makeatother
% Scale images if necessary, so that they will not overflow the page
% margins by default, and it is still possible to overwrite the defaults
% using explicit options in \includegraphics[width, height, ...]{}
\setkeys{Gin}{width=\maxwidth,height=\maxheight,keepaspectratio}
% Set default figure placement to htbp
\makeatletter
\def\fps@figure{htbp}
\makeatother
\setlength{\emergencystretch}{3em} % prevent overfull lines
\providecommand{\tightlist}{%
  \setlength{\itemsep}{0pt}\setlength{\parskip}{0pt}}
\setcounter{secnumdepth}{-\maxdimen} % remove section numbering
\ifLuaTeX
  \usepackage{selnolig}  % disable illegal ligatures
\fi
\IfFileExists{bookmark.sty}{\usepackage{bookmark}}{\usepackage{hyperref}}
\IfFileExists{xurl.sty}{\usepackage{xurl}}{} % add URL line breaks if available
\urlstyle{same}
\hypersetup{
  pdftitle={Taller\textbar{}},
  hidelinks,
  pdfcreator={LaTeX via pandoc}}

\title{Taller\textbar{}}
\author{}
\date{\vspace{-2.5em}2024-03-06}

\begin{document}
\maketitle

\hypertarget{taller-1}{%
\subsection{Taller 1}\label{taller-1}}

\begin{enumerate}
\def\labelenumi{(\arabic{enumi})}
\tightlist
\item
  ¿Hay multicolinealidad en los datos? Explique sucintamente.
\end{enumerate}

Verificar la multicolinealidad

El error en R al calcular el VIF (Variance Inflation Factor) indica que
hay coeficientes aliased (coeficientes aliados) en el modelo de
regresión. La multicolinealidad perfecta entre variables predictoras
puede resultar en coeficientes aliased, lo que causa problemas en
cálculos como el VIF.

\begin{enumerate}
\def\labelenumi{(\arabic{enumi})}
\setcounter{enumi}{1}
\tightlist
\item
  Separe aleatoriamente (pero guarde la semilla) su conjunto de datos en
  dos partes
\end{enumerate}

Dividir los df

\begin{enumerate}
\def\labelenumi{\arabic{enumi})}
\setcounter{enumi}{2}
\tightlist
\item
  Usando los 1000 datos de entrenamiento, determine los valores de λr y
  λl de regesión ridge y lasso, respectivamente, que minimicen el error
  cuadrático medio (ECM) mediante validación externa. Utilice el método
  de validación externa que considere más apropiado.
\end{enumerate}

\begin{enumerate}
\def\labelenumi{(\arabic{enumi})}
\setcounter{enumi}{3}
\tightlist
\item
  Ajuste la regresión ridge y lasso con los valores estimados de λr y λl
  obtenidos en (3) usando los 1000 datos de entrenamiento.
\end{enumerate}

\begin{enumerate}
\def\labelenumi{\arabic{enumi})}
\setcounter{enumi}{4}
\tightlist
\item
  Para los modelos ajustados en (4) determine el más apropiado para
  propósitos de predicción. Considere unicamente el ECM en los 200 datos
  de prueba para su decisión. Seleccionar el mejor modelo
\end{enumerate}

6. Ajuste el modelo seleccionado en (5) para los 1200 datos. Note que en
este punto ya tiene un λ estimado y un modelo seleccionado.

\end{document}
